\documentclass[11pt, letterpaper, oneside, titlepage]{article}
%\usepackage{fontspec}
%  \setmainfont{Hack}
\usepackage{geometry}
  \geometry{left=2cm, top=2.5cm, right=2cm, bottom=2cm} % Configure page margins with geometry
\usepackage{listings}
\usepackage{color}

%%%%%%%%%%%%%%%%%%%%%%%% Useful Macros %%%%%%%%%%%%%%%%%%%%%%%%%%%%%%%%%%%%%%%%%
%\newcommand{\commandLine}[1]{
%  \lstset{language=bash}
%  \begin{lstlisting}[frame=single]^^n
%    #1^^n
%  \end{lstlisting}
%}

\title{*nix Documentation \\\large Linux Hardeneing 101}
\author{ Xavier Loose \\ xav.loose@gmail.com \\ github.com/xavloose \and Zhixiu Lu \\ Zhixiu.Lu@coyotes.usd.edu }
\date{\today}

\begin{document}
% ./reivaX1!69$

  \maketitle
  %\commandLine{wget pussy}
  %-----------------------------------------------------------------------------
  % This is where text is imported. linuxSec/IntroTemplate.tex can be used as
  %   guide for developing a section
  %
  % Adding a section should follow the following format
  % \part{Name of Section}
  %   \input{relative/file/location/filename.tex}
  %   \input{relative/file/location/filename2.tex}
  %   ...
  %   \newpage
  %-----------------------------------------------------------------------------
  \part{Introduction}
    %%%%%%%%%%%%%%%%%%%%%%%%%%%%%%%%%%%%%%%%%%%%%%%%%%%%%%%%%%%%%%%%%%%%%%%%%%%%%%%%%
% Author: Xavier Loose
% 2017
%
% This doubles as a template for creating a section for USD's CCDC Competetion
%
%%%%%%%%%%%%%%%%%%%%%%%%%%%%%%%%%%%%%%%%%%%%%%%%%%%%%%%%%%%%%%%%%%%%%%%%%%%%%%%%

% The section is defined in main.tex so we only need to create subsections
% There is no need to format becuase we have already done that.

% These represent the heiarchy for sectioning a larger section
% If you keep proper indentation it will make it easier to follow for easier for
% future members
% \subsection{name}
% \subsubsection{name}
% \paragraph{name}
% \subparagraph{name}

\section{Preface}
  This section is primarily inteded for Linux since that is what we will most likely be dealing with at CCDC, but it should be noted that given another Unix system should not cause panic as I'll describe. Linux was a fork of the Unix kernel and since then has grown in popularity and has become more common than it's parent. Since our primary goal is to stick to the basics and a lot of the software is compatible with all *nix systems then. Linux specialites will be prefaced with "Linux" and "Unix" for unix systems.

\section{How to Use This}
  This documentation is intended to keep the information short and sweet for the purposes of competetion. The effort has been applied so that it can be quickly analyzed in a highly stressful environment.

  \subsection{Identifying Commands}
  

    \section{Basic}
  The first thing that we will be doing is logining as root and changing the password. This is as simple as follows:

  \$ su
  \$ passwd

  The next thing would be to mark every user as expired. This can be quickly done with the chage command which will also make the user incapable of loggin in as ssh. You want to view the /etc/passwd file to see who the users are.

  \$ cat /etc/passwd
  \$ chage -E 0 username

  Next install the cusom kernel that has been prebuilt. Debian-like systems will use the .deb packages and RHEL-like will use the *.rpm. The following commands will show how to install them.

  \$ dpkg -i *.deb
  \$ yum install *.rpm

  The kernels are in the software packages directory. The kernel is patched with grsecurity. In order to utilize the extra functionality you will need to download the pax tools from the grsecurity.org website and install. After this basic hardening should be done.

\section{Docker}
  \$ yum install -y yum-utils
  \$ yum-config-manager --add-repo https://download.docker.com/linux/centos/docker-ce.repo
  \$ yum-config-manager --enable docker-ce-edge
  \$ yum makecache fast
  \$ yum install -y docker-ce

    \newpage
  %\part{Phase 1}
  %  \input{linuxSec/Stage1.tex}
  %  \newpage
  %\part{Phase 2}
  %  \input{linuxSec/Stage2.tex}
  %  \newpage
  %\part{Phase 3}
  %  \input{linuxSec/Stage3.tex}
  %  \newpage
  %\part{Linux Kernel}
  %  \input{linuxSec/KernelSecurity.tex}
  %  \input{linuxSec/InstallingAKernel.tex}
  %  \input{linuxSec/BuildingAKernel.tex}
  %  \newpage
  %\part{Linux GRSecurity}
    %\newpage
  %\part{Linux SELinux}
    %\newpage
  %\part{ModSecurity}
  %  \input{linuxSec/ModSecurity.tex}
  %  \newpage
\end{document}
